\documentclass{article}
\usepackage[utf8]{inputenc}
\usepackage[margin=1in]{geometry}


\title{Project writeup (0.5 pages, including data source) \\
\large BST 227 - Machine Learning for Genomics}
\author{Markham Anderson}
\date{April 2019}

\begin{document}

\maketitle

% \LARGE
% Project writeup (0.5 pages, including data source) BST 227 - Machine Learning for Genomics \\
% Markham Anderson

\normalsize
My project is an unsupervised approach to identifying patterns of cell types in a 2D layout. It has come to my attention that in the last one or two years, datasets have become available which represent not only the genomic data of sampled cells but also the cells' location in a 2D surface. I anticipate being directed to one such dataset by Gerald Quon. I don't in fact expect to be able to complete an investigation of a given dataset with my limited computing resources and time, so I intend to actually use only a subset of the given dataset, and I intend to reduce its dimensionality by running PCA on the genomic profile data.

The aforementioned representation comprising both spatial and genomic information will enable me to construct a fully-connected graph, with the edge weight between cell $i$ and cell $j$ calculated as $d(i,j)\times c(i,j)$, where $d$ represents the spatial distance and $c$ represents the feature distance between cells $i$ and $j$. (For these two distances, I expect to work with Gaussian distance, but given time, I may also experiment with cosine distance or Euclidean distance for either one. Furthermore, I might experiment with variants of the distance calculation stated above so as to favour spatial and feature distance differently.)

To sparsify the graph, I will run $k$-nearest neighbours and keep only the edges for these neighbourhoods. I intend to train a graph convolution network (GCN) as an autoencoder on the graph data. Being that $k$-means will give me a fixed neighbourhood size, my GCN's aggregator function can be to simply concatenate genomic features of all of the cells in each neighbourhood. (This has shortcomings, but GCN's are new to me, so I intend this to be at least my first approach. If all proceeds quickly, I may experiment with other aggregator functions.) The embedding produced by the GCN's encoder will then serve as a new feature representation (replacing the former spatial and feature data) for each cell (or, rather, each neighbourhood, with a neighbourhood being focused on a single cell).

In this new feature space, I will perform spectral clustering. I intend to use the normalized Laplacian to this end, but given time, I may also experiment with a random-walk Lalacian. Moreover, I may experiment with clustering on the feature space in addition to clustering on the eigenvectors of the Laplacians. (GCN's and spectral clustering are both new to me, so I anticipate the possibility that I will not move quickly enough to compare results with alternative designs.)

The above-described clustering will give me sets of neighbourhoods. Within these sets, I will look for patterns of cell types. In a given mesh of $n$ cells, I expect to discover $k << n$ cell types. Unless I discover an obvious answer to the question of identifying cell types between now and the start of the project, I will endeavour to discover these cell types themselves by running spectral clustering on the cells' genomic profiles without any spatial data. For the sake of completing the work on time, I intend to do this step without a GCN, but if all proceeds quickly, I might train another GCN autoencoder to enhance this step of the project.

Having produced clusters of neighbourhoods and clusters of cell types, I will perform qualitative examination of neighbourhoods nearest to the cluster centroids, looking for cell-type consistency within clusters and differentiation between clusters.

For the upcoming baseline/strawman approach, I plan to apply spectral clustering directly to the graph data (omitting the aforementioned GCN).

\end{document}
